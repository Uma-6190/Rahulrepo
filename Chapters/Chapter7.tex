% Chapter Template

\chapter{Conclusion} % Main chapter title

\label{Chapter7} % Change X to a consecutive number; for referencing this chapter elsewhere, use \ref{ChapterX}

\lhead{Chapter 7. \emph{Conclusion}} % Change X to a consecutive number; this is for the header on each page - perhaps a shortened title

%----------------------------------------------------------------------------------------
%	SECTION 1
%----------------------------------------------------------------------------------------

\section{Summary of Important Points}
Through the course of the experiments shown in \autoref{Chapter4}, \autoref{Chapter5} and \autoref{Chapter6}, I have demonstrated that  \ac{MRL} can be used to learn a grounded representation of different data modalities in an unsupervised manner.

This representation has been demonstrated to fit the criteria layed out by Bengio et al. in \cite{repRev}. Particularly, the representation has been demonstrated to have Manifolds which can be manipulated through vector arithmetic to make predictable and meaningful changes in the output.

In some datasets, such as MNIST and ArtS, it is possible to generate image prototypes for each class/word in the \ac{MAE}'s vocabulary. For MNIST, this came in the form of prototypical versions of each digit and for ArtS, images of each colour, shape and size can be generated individually.

This is also possible for the ReShape dataset However, due to the larger variability of training examples, the prototypes are very blurry. By using a class exemplar for each object, the \ac{MAE} was able to learn to generate non-blurry prototypes for each of the visual attributes.

I also demonstrated that \ac{MRL} can be used to generate accurate object descriptions of both real and artificial images. As well as that using \ac{MRL} can led to improvements in classification accuracy as seen in \autoref{Chapter4}.


\section{Conclusion}
\ac{MRL} has been shown to be a powerful technique and presents an area worthy of futher study. Whilst the findings of the experiments in this thesis are exciting, further work is needed to apply \ac{MRL} in a real world setting.

